\documentclass[ucs,10pt]{beamer}

% Copyright 2004 by Till Tantau <tantau@users.sourceforge.net>.
%
% In principle, this file can be redistributed and/or modified under
% the terms of the GNU Public License, version 2.
%
% However, this file is supposed to be a template to be modified
% for your own needs. For this reason, if you use this file as a
% template and not specifically distribute it as part of a another
% package/program, I grant the extra permission to freely copy and
% modify this file as you see fit and even to delete this copyright
% notice.
%
% Modified by Tobias G. Pfeiffer <tobias.pfeiffer@math.fu-berlin.de>
% to show usage of some features specific to the FU Berlin template.

% remove this line and the "ucs" option to the documentclass when your editor is not utf8-capable
\usepackage[utf8x]{inputenc}    % to make utf-8 input possible
\usepackage[ngerman]{babel}     % hyphenation etc., alternatively use 'german' as parameter

\include{fu-beamer-template}  % THIS is the line that includes the FU template!

\usepackage{arev,t1enc} % looks nicer than the standard sans-serif font
% if you experience problems, comment out the line above and change
% the documentclass option "9pt" to "10pt"

% image to be shown on the title page (without file extension, should be pdf or png)
\titleimage{fu_500}

\title[Milestone 1] % (optional, use only with long paper titles)
{Milestone 1}

\subtitle
{Vorstellung der Gruppe \emph{fuc} und des aktuellen Standes}

\author[Pilz] % (optional, use only with lots of authors)
{Sven-Kristofer Pilz}
% - Give the names in the same order as the appear in the paper.

\institute[FU Berlin] % (optional, but mostly needed)
{Freie Universität Berlin}
% - Keep it simple, no one is interested in your street address.

\date[SWPCOM 2012] % (optional, should be abbreviation of conference name)
{Softwareprojekt Übersetzerbau}
% - Either use conference name or its abbreviation.
% - Not really informative to the audience, more for people (including
%   yourself) who are reading the slides online

\subject{Milestone 1}
% This is only inserted into the PDF information catalog. Can be left
% out.

% you can redefine the text shown in the footline. use a combination of
% \insertshortauthor, \insertshortinstitute, \insertshorttitle, \insertshortdate, ...
\renewcommand{\footlinetext}{\insertshortinstitute, \insertshorttitle, \insertshortdate}

% Delete this, if you do not want the table of contents to pop up at
% the beginning of each subsection:
\AtBeginSection[]
{
  \begin{frame}<beamer>{Outline}
    \tableofcontents[currentsection]
  \end{frame}
}

\newlength{\wideitemsep}
\setlength{\wideitemsep}{\itemsep}
\addtolength{\wideitemsep}{5pt}
\let\olditem\item
\renewcommand{\item}{\setlength{\itemsep}{\wideitemsep}\olditem}

\setbeamertemplate{itemize item}[circle]
\setbeamertemplate{itemize subitem}[triangle]

\begin{document}

\begin{frame}[plain]
  \titlepage
\end{frame}

\begin{frame}{Inhalt}
  \tableofcontents
  % You might wish to add the option [pausesections]
\end{frame}

\section{Organisation}

\subsection{Arbeitsgruppen}
\begin{frame}{Arbeitsgruppen mit Ansprechpartner}
  \begin{itemize}
  \item Lexer
  \begin{itemize}
  	\item Thomas.
  \end{itemize}
  
  \item Parser
  \begin{itemize}
	\item Björn und Samuel.
  \end{itemize}
  
  \item Semantische Analyse
  \begin{itemize}
  	\item Christoph, Eduard und Sven.
  \end{itemize}    
  
  \item Drei-Adress-Code
  \begin{itemize}
  	\item Frank, Danny und Manuel.
  \end{itemize}    
  
  \item LLVM-Backend
  \begin{itemize}
  	\item Roman, Moritz, Jens
  \end{itemize}
  \end{itemize}
\end{frame}

\subsection{Kommunikation}
\begin{frame}{Kommunikation}
  \begin{itemize}
  	\item Arbeitsgruppen organisieren interne Kommunikation eigenständig.
  	\begin{itemize}
  		\item Detailfragen entscheidet die Arbeitsgruppe.
  	\end{itemize}
  	\item Jeder spricht mit jedem.
  	\item Eigene Mailingliste.
  	\item Jeden Donnerstag Treffen der gesamten Gruppe.
  	\begin{itemize}
  		\item Jede Arbeitsgruppe berichtet Status.
  		\item Größere Entscheidungen per Abstimmung.
  	\end{itemize}
  	\item \dots für alles andere gibt es den Projektleiter.
  \end{itemize}
\end{frame}

\subsection{Entwicklung}
\begin{frame}{Entwicklung}
  \begin{itemize}
  	\item Einzige Regel: Master Branch muss immer lauffähig sein.
  	\item Jeder darf in Master \emph{pushen}.
  	\item Arbeitsgruppen arbeiten in eigenen Branches.
  	\item Test Driven
  	\begin{itemize}
  		\item Tests werden per ANT ausgeführt, nutzen CI von GitHub.
  	\end{itemize}
  \end{itemize}
\end{frame}

\section{Milestone 1}
\subsection{Lexer}
\begin{frame}{Lexer}
	\begin{itemize}
		\item Sprachumfang für Milestone 1 implementiert.
		\item Quelltext wird aus InputStream als Liste von Zeilen eingelesen.
		\item Token als reguläre Ausdrücke spezifiziert.
		\begin{itemize}
			\item Bereits gesamter Sprachumfang spezifiziert.
		\end{itemize}
	\end{itemize}
\end{frame}

\subsection{Parser}
\begin{frame}{Parser}
	work in progress
	
	\pause
	
	\begin{center}
	\includegraphics[width=0.7\textwidth]{parser}
	\end{center}
\end{frame}

\subsection{Semantische Analyse}
\begin{frame}[fragile]{Semantische Analyse}
	\begin{itemize}
		\item Fehler aus Sprachumfang für Milestone 1 implementiert.
		\item Findet folgenden Fehler:
		\begin{itemize}
			\item Verwendung von Variablen ohne Initialisierung.
				\begin{lstlisting}[language=C]
long i
return i
				\end{lstlisting}
		\end{itemize}
	\end{itemize}
	

\end{frame}

\subsection{Drei-Adress-Code}
\begin{frame}{Drei-Adress-Code}
	\begin{itemize}
		\item Sprachumfang für Milestone 1 implementiert.
		\item Weiterer Sprachumfang bereits als Stubs vorhanden.
		\item Hat auch den AST implementiert.
		\begin{itemize}
			\item Testabdeckung von etwa 100\%.
		\end{itemize}
		\item \dots wohl bereits sehr beliebt.
	\end{itemize}
\end{frame}

\subsection{LLVM-Backend}
\begin{frame}{LLVM-Backend}
	\begin{itemize}
		\item Drei-Adress-Code für Milestone 1 ist implementiert.
		\item Kann den Drei-Adress-Code auch aus Textdatei lesen.
		\item Erzeugter LLVM-Code ließ sich erfolgreich testen.
		\item Kann Drei-Adress-Code auch direkt ausführen und Rückgabewert prüfen.
		\begin{itemize}
			\item Dazu ist LLVM auf dem System erforderlich.
		\end{itemize}
	\end{itemize}
\end{frame}

\subsection{Zusammenfassung}
\begin{frame}{Zusammenfassung (Milestone 1)}
	\begin{itemize}
	\item Fertig (samt Tests):
	\begin{itemize}
		\item Lexer
		\item Semantische Analyse
		\item Drei-Adress-Code
		\item LLVM-Backend
	\end{itemize}
	
	\item In Arbeit:
	\begin{itemize}
		\item Parser
		\item Visualisierung des Drei-Adress-Codes
		\item Visualisierung des Token-Stream
		\item Integration und Controller
	\end{itemize}
	\end{itemize}
\end{frame}

\begin{frame}{Auf Wiedersehen}
	Fragen?	
\end{frame}
\end{document}